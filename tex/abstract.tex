\begin{quote}
    \enquote{... Do one thing well is usually interpreted as being about simplicity. But it's also, implicitly and at least as importantly, about orthogonality ...}~\cite{orthogonality}.
\end{quote}


\begin{abstract}

% Access, Automation, Analytics -> AI

Unrestricted \emph{Access} to algo and data is essential to augment the limited cognitive capabilities of humans, enabling the construction of vast systems that expand beyond the horizon. These large-scale systems are often developed by individuals who may never meet or might even be antagonistic due to local office dynamics. The next desires are \emph{Automation}, to enable linear descriptions of exponentially-expanding complexity, and \emph{Analytics}, to compress the resulting volumes of information into the bandwidth of human operators for effective consumption.

This paper demonstrates that Access, Automation, and Analytics are achieved through custom languages that are built for the context based on a mathematical model. This model, for the sake of demonstration, is comparable to fitting the spanning tree of an Application Programming Interface (API) to a dataset of use of digital tools by humans recorded as sequences of keystrokes and mouse clicks. We propose a mathematical framework for discussing this category of languages and provide a reference implementation~\cite{abcli} based on \emph{Bash}~\cite{gnu_bash} expansions that call into \emph{Python}~\cite{python}. We note that Access, Automation, and Analytics have been suggested as the precursors of AI~\cite{Slife24}.

We use AWS SageMaker~\cite{sagemaker} for development and training and AWS Batch~\cite{aws_batch} for inference and compute. 

\footnote{built by \texttt{gizai-\revision}.}
\end{abstract}