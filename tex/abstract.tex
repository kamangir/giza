\begin{quote}
    \enquote{... Do one thing well is usually interpreted as being about simplicity. But it's also, implicitly and at least as importantly, about orthogonality ...}~\cite{orthogonality}.
\end{quote}


\begin{abstract}
Mathematical optimization of \emph{Access} to data can yield primitive standards for basic AI needs such as data provenance through named objects with queryable metadata through an \emph{MLflow}~\cite{mlflow}--style backend. Similarly, on the algo side, AI operations can be modelled as \emph{Bash}~\cite{gnu_bash} commands, to enable state-of-the-art \emph{Automation} through job-processing systems such as \emph{AWS Batch}~\cite{aws_batch} and Argo Workflows~\cite{argoflow}. The same object metadata mechanisms carry the \emph{Analytics} that produce a compressed view of search spaces with exponentially expanding dimensions. Thus, Analytics provides Access to the fruits of Automation.

This paper proposes a mathematical framework to achieve Access, Automation, and Analytics through building a custom language for the context. This model, for the sake of demonstration, is comparable to fitting the spanning tree of an Application Programming Interface (API)~\cite{fielding2000rest} to a dataset of use of digital tools by operators recorded as sequences of keystrokes and mouse clicks. We propose a mathematical framework for discussing this category of languages and provide a reference implementation~\cite{abcli} based on Bash expansions that call into \emph{Python}~\cite{python}. We note that Access, Automation, and Analytics have been suggested as the precursors of AI~\cite{Slife24}~\footnote{built by \texttt{gizai-\revision}.}.
\end{abstract}