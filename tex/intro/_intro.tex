\section{Introduction}
\label{intro}

A shell is a \enquote{macro processor that executes commands}~\footurl{https://www.gnu.org/software/bash/manual/html_node/What-is-a-shell_003f.html}, where \enquote{macro processor means functionality where text and symbols are expanded to create larger expressions} (same reference). There are seven kinds of expansions in bash~\footurl{https://www.gnu.org/software/bash/manual/html_node/Shell-Expansions.html}. \emph{Brace Expansion}~\footurl{https://www.gnu.org/software/bash/manual/html_node/Brace-Expansion.html} is the first and the quickest to explain,
%
\begin{verbatim}
 > bash$ echo a{d,c,b}e
ade ace abe
\end{verbatim}
%
\emph{Tilde Expansion}~\footurl{https://www.gnu.org/software/bash/manual/html_node/Tilde-Expansion.html} relates to words that begin with an unquoted tilde character (\textasciitilde). \emph{Parameter and Variable Expansion}~\footurl{https://www.gnu.org/software/bash/manual/html_node/Shell-Parameter-Expansion.html} enable the use of variables, as \texttt{\$\{variable\}}, as well as more elaborate pattern matching forms such as \texttt{\$\{parameter/\#pattern/string\}}. \emph{Command Substitution} \enquote{allows the output of a command to replace the command itself} ~\footurl{https://www.gnu.org/software/bash/manual/html_node/Command-Substitution.html}. \emph{Arithmetic expansion}~\footurl{https://www.gnu.org/software/bash/manual/html_node/Arithmetic-Expansion.html} enables arithmetic operations using the form \texttt{\$(( expression ))
} and \emph{Word Splitting}\footurl{https://www.gnu.org/software/bash/manual/html_node/Word-Splitting.html} governs the splitting of the command to words. Finally, \emph{Filename Expansion}~\footurl{https://www.gnu.org/software/bash/manual/html_node/Filename-Expansion.html} enables the familiar wildcard reference to filenames using `*' and `?'. In Section~\ref{expansions} we propose a set of expansions that are relevant to AI operations.