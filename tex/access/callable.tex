\marginpar{Callable Expansion}
When the callable \keyword{<func>} receives a command that starts with the task \keyword{<task>}, it calls the function \keyword{<func>\_<task>} with the rest of the command, if such function exists. For example, here, the callable \keyword{bluer\_ai\_conda}, which is aliased to \keyword{@conda}, is defined.

\begin{graybox}
\begin{verbatim}
#! /usr/bin/env bash

function bluer_ai_conda() {
    local task=$1

    local function_name=bluer_ai_conda_$task
    if [[ $(type -t $function_name) == "function" ]]; then
        $function_name "${@:2}"
        return
    fi

    conda "$@"
}

bluer_ai_source_caller_suffix_path /conda
\end{verbatim}
\end{graybox}
%
This mechanism works with the function \keyword{bluer\_ai\_conda\_exists}, which resides in \keyword{/conda/exists.sh}, listed below,
%
\begin{graybox}
\begin{verbatim}
#! /usr/bin/env bash

function bluer_ai_conda_exists() {
    local options=$1
    local environment_name=$(bluer_ai_option "$options" name bluer_ai)

    if conda info --envs | grep -q "^$environment_name "; then
        echo 1
    else
        echo 0
    fi
}    
\end{verbatim}
\end{graybox}
%
The result is the \emph{super-command} \keyword{@conda} which behaves as \keyword{conda}, except when the function \keyword{bluer\_ai\_conda\_<task>} is defined. Here is an example of the outcome assembled together,
%
\begin{graybox}
\begin{verbatim}
[[ $(@conda exists name=<env-name>) == 1 ]] && echo "found."
\end{verbatim}
\end{graybox}
%
We use this \emph{namespacing}~\cite{namespaces} mechanism for organization as well as orchestration. In practice, this expansion yields callables with multiple prefixes. We generate \keyword{@<keyword>} aliases~\cite{aliases} to facilitate user access, as stated above. In practice, the use of this expansion is comparable to fitting the spanning tree of an Application Programming Interface (API)~\cite{fielding2000rest} to a dataset of use of digital tools by operators recorded as sequences of keystrokes and mouse clicks.