\marginpar{Objects}
Commands manipulate data as objects. An object is a uniquely named collection of files and folders and can be downloaded or uploaded, in part or as a whole. Some objects already exist in the environment. For example, an \emph{item} in a \emph{STAC collection}~\cite{stac_intro_tutorial} (a datacube) or a dataset in \emph{Kaggle}~\cite{chen2019-AIRD-dataset} are objects. A curated dataset, a model trained on it, and the model's predictions on a datacube, are examples of other objects.

\marginpar{Object Pointers}
An object may be \emph{selected},
%
\begin{verbatim}
@select <object-name>
\end{verbatim}
%
When \keyword{<object-name>} is selected,  `\keyword{.}' expands to \keyword{<object-name>}. Similarly, `\keyword{..}', `\keyword{...}', and so on, as deep as needed, expand to the names of the previously selected object and the one before that. Commands default the objects they consume and modify to `\keyword{.}', `\keyword{..}', and so on. Therefore, because the commands in a script generally use the same objects, selecting the objects enables their names to be replaced with pointers. Often the defaults of the commands are designed to enable the omission of the pointers as well.

\marginpar{Object Metadata}
\emph{Metadata} is information about objects, such as their tags, maintained in a system such as \emph{MLflow}~\cite{mlflow}. Objects also carry a dictionary named \keyword{metadata.yaml}. One can \keyword{get} and \keyword{post} the metadata corresponding to an object.

\marginpar{Object Tags}
An object can have many tags. A tag is a boolean or valued property of the object and is \keyword{set} and \keyword{get}, and can be \keyword{search}ed,
%
\begin{verbatim}
@mlflow tags get [.|<object-name>] [--tag <tag>]

@mlflow tags search [<keyword-1>=<value-1>,<keyword-2>,~<keyword-3>]

@mlflow tags set [.|<object-name>] [<keyword-1>=<value>,<keyword-2>,~<keyword-3>]
\end{verbatim}

\marginpar{Object Persistence}
Object are persisted on AWS S3~\cite{aws_s3}. Objects can be downloaded, uploaded, and listed,
%
\begin{verbatim}
@download [filename=<filename>] [.|<object-name>]

@upload [filename=<filename>] [.|<object-name>]
    
@list cloud|local [.|<object-name>]
\end{verbatim}