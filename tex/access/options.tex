\subsection{Options}

% edited up to here - 2025-03-01

An options is a string representation of a dictionary, such as,
%
\begin{verbatim}
<keyword-1>=<value-1>,<keyword-2>=<value-2>,...,<keyword-3>,-<keyword-4>},...
\end{verbatim}
%
Options is implemented using basic Python and, therefore, the \emph{options expansion} is available to Bash commands through command substitution~\cite{command_substitution}. In practice, the two additional expansions \keyword{@option::bool} and \keyword{@option::choice} cast the output to boolean and select it from a list (equivalent to an Enum~\cite{python_eunum}), respectively.
%
\begin{verbatim}
value=$(@option "$options" <keyword> [<default>])

value=$(@option::bool "$options" <keyword> 0 | 1)

value=$(@option::choice "$options" <comma,separated,list> <default>)
\end{verbatim}
