\subsubsection{@cache}
\label{cache}

The cache is a dictionary available on every machine~\ref{machine} for read, \texttt{@cache read <keyword>}, write, \texttt{@cache write <keyword> <value>}, and search, \texttt{@cache search <keyword>} that is enabled either through a SQL database~\footurl{https://github.com/kamangir/awesome-bash-cli/blob/main/abcli/plugins/tags/functions.py} or a tool such as \texttt{mlflow}~\footurl{https://mlflow.org/}.

When used for objects~\ref{objects}, \texttt{@cache} provides a tagging~\ref{tag} mechanism,
%
\begin{verbatim}
@cache read <object-name>.<keyword>
@cache write <object-name>.<keyword> <value>
\end{verbatim}
%
given that \texttt{@cache clone <object-1> <object-2>} is supported. 

\subsubsection{@ref}
\label{ref}

\texttt{@ref} is an alias for \texttt{@cache read} that enables \texttt{\$(@ref <keyword>)} and, thus, object pointers~\ref{pointers}.