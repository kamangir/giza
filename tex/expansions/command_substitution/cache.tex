\subsubsection{@cache}
\label{cache}

The cache is a dictionary available on every machine~\ref{machine} for read, \texttt{@cache read <keyword>}, write, \texttt{@cache write <keyword> <value>}, and search, \texttt{@cache search <keyword>}. It is of specific interest to cache properties of objects~\ref{objects},
%
\begin{verbatim}
@cache read <object-name>.<keyword>
@cache write <object-name>.<keyword> <value>
\end{verbatim}
%
which are used in conjunction with \texttt{@cache clone <object-1> <object-2>}.

In addition to enabling \texttt{\$(@ref <keyword>)}~\ref{ref}, \texttt{@cache } is usable in for loops,
%
\begin{verbatim}
    local object_name
    for object_name in $(@cache search published --delim space); do
        <command> $object_name <args>
    done
\end{verbatim}
%
This example uses the \texttt{--delim}~\ref{delim} convention.

\subsubsection{@ref}
\label{ref}

\texttt{@ref} is an alias for \texttt{@cache read} that enables \texttt{\$(@ref <keyword>)} and, thus, object pointers~\ref{pointers}.