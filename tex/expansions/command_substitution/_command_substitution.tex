\subsection{Command Substitution}\label{command_substitution}

During the execution of the command \texttt{<part-1>\$(<sub-command>)<part-2>}, \texttt{<sub-command>} is executed and its outcome, \texttt{<outcome>}, is used to generate the updated command as \texttt{<part-1><outcome><part-2>}, which is then executed~\cite{command_substitution}. Here is an example from \keyword{roofAI},
%
\begin{verbatim}
roofAI semseg predict \
    profile=FULL,upload \
    $(@ref roofAI_semseg_model_AIRS_full_v2) \
    $(@ref roofAI_ingest_AIRS_v2) \
    $(@timestamp)
\end{verbatim}
%
Here, \texttt{@ref <keyword>} reads the value of \texttt{<keyword>} from the \keyword{cache} and \texttt{@timestamp} generates a unique timestamp for use as an \texttt{<object-name>}. Collectively, this command runs the \enquote{Pytorch Segmentation Model}~\cite{smp} that is cached as \texttt{roofAI\_semseg\_model\_AIRS\_full\_v2} on the dataset that is cached as \texttt{roofAI\_ingest\_AIRS\_v2} and uploads the results in a uniquely named \keyword{objects}. \keywordd{Tags}{tag} and \keyword{relations} are other object metadata relevant to this expansion.

Command substitution is useful for generating the command components through Python or Bash. For example, in the AWS Open Data Registry~\cite{aws_open_data} there is the notion of datasets, such as \texttt{hst}~\cite{hst} for Hubble Space Telescope and the metadata the dataset is maintained in \texttt{yaml} files a git repository~\footnote{\texttt{datasets/hst.yaml}~\cite{aws_open_data}.}. 

For example, here is the command to access \texttt{ibrma2f2q\_drc.jpg} in object \texttt{public/ibrm/ibrma2f2q} in the dataset \texttt{hst},
%
\begin{verbatim}
aws s3 cp $auth $s3_uri$filename $path
\end{verbatim}
%
Here, \texttt{\$auth} and \texttt{\$s3\_uri} are generated as,
%
\begin{verbatim}
auth=$(hubble_get auth $dataset_name)
s3_uri=$(hubble_get s3_uri $dataset_name $hubble_object_name)
\end{verbatim}
%
Here, \texttt{abcli\_hubble\_get} is a Bash wrapper around a Python call.
%
\begin{verbatim}
function hubble_get() {
    python3 -m hubblescope get \
        --what "$1" \
        --dataset_name "$2" \
        --object_name "$3" \
        "${@:4}"
}
\end{verbatim}

\subsubsection{@cache}
\label{cache}

The cache is a dictionary available on every machine~\ref{machine} for read, \texttt{@cache read <keyword>}, write, \texttt{@cache write <keyword> <value>}, and search, \texttt{@cache search <keyword>}. It is of specific interest to cache properties of objects~\ref{objects},
%
\begin{verbatim}
@cache read <object-name>.<keyword>
@cache write <object-name>.<keyword> <value>
\end{verbatim}
%
which are used in conjunction with \texttt{@cache clone <object-1> <object-2>}.

In addition to enabling \texttt{\$(@ref <keyword>)}~\ref{ref}, \texttt{@cache } is usable in for loops,
%
\begin{verbatim}
    local object_name
    for object_name in $(@cache search published --delim space); do
        <command> $object_name <args>
    done
\end{verbatim}
%
This example uses the \texttt{--delim}~\ref{delim} convention.

\subsubsection{@ref}
\label{ref}

\texttt{@ref} is an alias for \texttt{@cache read} that enables \texttt{\$(@ref <keyword>)} and, thus, object pointers~\ref{pointers}.
\subsubsection{@tag}
\label{tag}

An object~\ref{objects}~\ref{objects} can have many tags. A tag is a boolean or valued property of the object and is \texttt{set} and \texttt{get}, and can be \texttt{search}ed,
%
\begin{verbatim}
@tags set <object-name> <options>

@tags get <object-name>
@tags get <object-name> <keyword>

@tags search <options>
\end{verbatim}
%
This expansion uses \texttt{options}~\ref{options}.
\subsubsection{@relations}
\label{relations}

Two objects can be related in many ways.

\begin{verbatim}
@relations get <object-name-1> <object-name-2>
@relations get <object-name-1> <object-name-2> <relation>
@relations get <object-name-1>
@relations get <object-name-1> <relation>

@relations search <relation>

@relations set <object-name-1> <object-name-2> <relation>
\end{verbatim}
