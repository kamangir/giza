\subsection{Command Substitution}
\label{command_substitution}

In general, the execution of \texttt{<part-1>\$(<sub-command>)<part-2>} waits for the execution of \texttt{<sub-command>}, which we assume yields \texttt{<output>}. Then, the command is updated to \texttt{<part-1><output><part-2>}.





This expansion is useful for generating a part of the command through Python or Bash. For example, in the AWS Open Data Registry~\footurl{https://registry.opendata.aws/} there is the notion of datasets, such as \texttt{hst}~\footurl{https://registry.opendata.aws/hst/} for Hubble Space Telescope and metadata about each dataset is maintained in a git repository~\footurl{https://github.com/awslabs/open-data-registry/blob/main/datasets/hst.yaml}. \texttt{hubble} is a $\nabla$ giza callable~\ref{callable} to access Hubble Space Telescope imagery and other datasets on AWS Open Data Registry~\footurl{https://github.com/kamangir/hubble}.




wip

, if \texttt{\$(<sub-command>)} is encountered, then \texttt{<sub-command>} is executed and its outcome is substituted in \texttt{<command>} and the execution continues~\footurl{https://www.gnu.org/software/bash/manual/html\_node/Command-Substitution.html}.

\subsubsection{@cache}
\label{cache}

The cache is a dictionary available on every machine~\ref{machine} for read, \texttt{@cache read <keyword>}, write, \texttt{@cache write <keyword> <value>}, and search, \texttt{@cache search <keyword>}. It is of specific interest to cache properties of objects~\ref{objects},
%
\begin{verbatim}
@cache read <object-name>.<keyword>
@cache write <object-name>.<keyword> <value>
\end{verbatim}
%
which are used in conjunction with \texttt{@cache clone <object-1> <object-2>}.

In addition to enabling \texttt{\$(@ref <keyword>)}~\ref{ref}, \texttt{@cache } is usable in for loops,
%
\begin{verbatim}
    local object_name
    for object_name in $(@cache search published --delim space); do
        <command> $object_name <args>
    done
\end{verbatim}
%
This example uses the \texttt{--delim}~\ref{delim} convention.

\subsubsection{@ref}
\label{ref}

\texttt{@ref} is an alias for \texttt{@cache read} that enables \texttt{\$(@ref <keyword>)} and, thus, object pointers~\ref{pointers}.
\subsection{\texttt{@tag}}
\label{tag}

An \keyword2{objects} can have many tags. A tag is a boolean or valued property of the object and is \texttt{set} and \texttt{get}, and can be \texttt{search}ed,
%
\begin{verbatim}
@tags set <object-name> <options>

@tags get <object-name>
@tags get <object-name> <keyword>

@tags search <options>
\end{verbatim}
\subsection{\texttt{@relations}}\label{relations}

Two \keyword{objects} can be related in several ways, each defined as a pair, to enable directional relations~\footabcli{abcli/plugins/relations/relations.json}. Here is an example,
%
\begin{verbatim}
{
    "added-to": "contains",
    "cloned": "cloned-by",
    ...
    "trained": "trained-on"
}
\end{verbatim}
%
\texttt{relations} can be \texttt{set}, \texttt{get}, and \texttt{search}ed, 
%
\begin{verbatim}
@relations set <object-name-1> <object-name-2> <relation>

@relations get <object-name-1> <object-name-2>
@relations get <object-name-1> <object-name-2> <relation>
@relations get <object-name-1>
@relations get <object-name-1> <relation>

@relation search <object-name> [--relation <relation>]
\end{verbatim}
