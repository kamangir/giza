\subsection{\texttt{options}}
\label{options}

An \texttt{options} is a string representation of a dictionary, such as,
%
\begin{verbatim}
<keyword-1>=<value-1>,<keyword-2>=<value-2>,...,<keyword-3>,-<keyword-4>},...
\end{verbatim}
%
\texttt{options} is implemented using basic Python~\footnote{\abcli{abcli/options}} and, therefore, the \emph{options expansion} is available to Bash commands through command substitution~\ref{command_substitution}~\footnote{In practice, a separate \texttt{@option::bool} is defined to cover boolean variables. It may be possible to combine \texttt{@option::bool} into \texttt{@option}, which remains an interest of the author.},
%
\begin{verbatim}
var=$(@option "$options" <keyword> <default>)
\end{verbatim}
%
Another useful expansion is the \emph{options choice expansion}~\footnote{
As an example, this expansion allows \texttt{abcli list cloud|local <object-name>} using \texttt{where=\$(option::choice "\$options" cloud,local cloud)}.
},
%
\begin{verbatim}
var=$(@option::choice "$options" <comma,separated,list> <default>)
\end{verbatim}
%
The following three expansions are also useful on \texttt{options}.
%
\begin{description}
%
\item[default]{\texttt{default} \texttt{<options-1>} to the corresponding values in \texttt{<options-2>}. The keyword set of the output is the concatenation of the keyword sets of the two inputs, wherein the values from \texttt{<options-1>} take priority. For example, defaulting \texttt{x=1,y=2} to \texttt{x=3,z=4} yields \texttt{x=1,y=2,z=4}. This is the \emph{default option expansion}, which is achieved through \texttt{<options-2>,<options-1>}.}
%
\item[subset]{return the \texttt{<options-1>} \texttt{subset} of \texttt{<options-2>}. The keyword set of the output is the same as the keyword set of \texttt{<options-1>}, wherein the values from \texttt{<options-2>} take priority. For example, the \texttt{x=1,y=2} subset of \texttt{x=3,z=4} yields \texttt{x=3,y=2}. This is the \emph{option subset expansion},
\begin{verbatim}
var=$(@option::subset <options-1> <options-2>)
\end{verbatim}
}
%
\item[update]{\texttt{update} \texttt{<options-1>} to \texttt{<options-2>}. The keyword set of the output is the concatenation of the keyword sets of the two inputs, wherein the values from \texttt{<options-1>} take priority. For example, updating \texttt{x=1,y=2} to \texttt{x=3,z=4} yields \texttt{x=3,y=2,z=4}. This is the \emph{option update expansion}, which is achieved through \texttt{<options-1>,<options-2>}.}
%
\end{description}
