\subsubsection{@select and object references}
\label{select}

\begin{verbatim}
@select <object-name>
@select <type> <typed-object-name>
\end{verbatim}
%
When an object is selected, \texttt{.} expands to \texttt{<object-name>}. Similarly, \texttt{..}, \texttt{...}, and so on, as deep as needed, expand to the previously selected object and the one before that. Commands default the objects they consume and modify to \texttt{.}, \texttt{..}, and so on. Because the commands in a script use the same objects, selecting the objects enables their names to be omitted in a script.
%
\begin{verbatim}
@select <object-name>
@download
<command-1> # . omitted for convenience
<command-2>
<command-3>
@upload
\end{verbatim}
%
An object may have a \emph{type}, for example, model or dataset. Commands that consume objects specify a type for the argument. This enables the user to simultaneously select different types of objects and run commands on them. Here is an example from \emph{hubble}~\footurl{https://github.com/kamangir/hubble}, wherein the user selects an object, then selects a hubble dataset,  then selects an object in that dataset and downloads it.
%
\begin{verbatim}
@select
hubble select dataset hst
hubble select object public/u4ge/u4ge0106r
hubble download -dryrun
\end{verbatim}