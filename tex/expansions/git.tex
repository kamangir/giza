\subsection{\texttt{@git}}
\label{git}

The \keyword{core} and the \keyword{plugins} are maintained in individual repositories. There are also other repositories of interest.
%
\begin{verbatim}
@git clone <repo-name> [init,install,object,<options>]
\end{verbatim}
%
clones \texttt{<repo-name>} and,
%
\begin{verbatim}
@git <repo-name>

@git cd|pushd <repo-name>
\end{verbatim}
%
change directory to \texttt{<repo-name>}.
%
\begin{verbatim}
@git <repo-name> <command>
\end{verbatim}
%
runs \texttt{<command>} in \texttt{<repo-name>}. The following expansions show the status of a repository, create a branch in it or push and pull from it. In the absence of \texttt{<repo-name>}, \texttt{status} and \texttt{pull} default to all repositories.
%
\begin{verbatim}
@git status [<repo-name>]

@git create_branch <branch-name> [.|<repo-name>]

@git pull [<repo-name>]

@git push [first,-status] [<message>] [.|<repo-name>]
\end{verbatim}