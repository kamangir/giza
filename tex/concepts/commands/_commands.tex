\subsection{Commands}
\label{commands}

A \emph{command} is any Bash command~\footurl{https://www.gnu.org/software/bash/manual/bash.html\#Shell-Syntax} and can be represented in a Python string of characters~\footurl{https://docs.python.org/3/library/string.html}. Here is an example command,
%
\begin{verbatim}
vanwatch ingest \
    vancouver \
    dryrun \
    . \
    --count 12
\end{verbatim}
%
The above command and the one below are \emph{identical}.
%
\begin{verbatim}
vanwatch discover vancouver ~upload --validate 1
\end{verbatim}
%
Two commands are identical if running them on two machines in identical states yields the same states. In theory, the state of any machine depends on the state of any other machine, and it is almost impossible to run two commands in identical states, including the time of execution. Therefore, when we refer to two identical commands, we either use a derivation-based proof of identity or consider a validation in a limited ``relevant'' subset of the state representation.

For any shell on any machine, at known states, there is a mapping between the set of all commands and $\{True,False\}$ that we address as ``whether the command is found''. In Bash, for example, the following message is printed when a command ``is not found''.
%
\begin{verbatim}
-bash: void: command not found
\end{verbatim}
%
Note that writing to the standard streams \emph{stdin} and \emph{stdout} are examples of state changes in the shell and the machine. 

\emph{Terraforming} is the process of running commands that modify the state of the shell and the machine in ways that make additional commands found. Terraforming is also intended to modify the state change caused by a set of commands favourable to the interests of an operator. For convenience, we address a command ``that is found'' as a \emph{valid} command and \emph{invalid} otherwise. Terraforming may also ensure the states of the machine and the shell after a restart. Terraforming generally includes a modification of \emph{bashrc}, \emph{bash\_profile}, \emph{screenrc}, and \emph{desktop} files.

Commands know the machine and the shell they are running in and adjust their operation accordingly. For example, a script that submits jobs inside a docker container may download the artifacts generated through previous submissions on a user-facing machine, such as a Macbook. 

While self-referential commands may be possible, in practice, commands follow a tree structure, wherein the execution of sub-commands contributes to the composition of the main command. This is because, during the execution of \texttt{<command>}, if \texttt{\$(<sub-command>)} is encountered, then \texttt{<sub-command>} is executed and its outcome is substituted in \texttt{<command>} and the execution continues~\footurl{https://www.gnu.org/software/bash/manual/html\_node/Command-Substitution.html}. Here is an example,
%
\begin{verbatim}
roofAI semseg predict \
    profile=FULL,upload \
    $(@ref roofAI_semseg_model_AIRS_full_v2) \
    $(@ref roofAI_ingest_AIRS_v2) \
    $(@timestamp)
\end{verbatim}
%
Here, \texttt{@ref <keyword>} reads the value of \texttt{<keyword>} from the \texttt{cache}~\ref{cache} and \texttt{@timestamp} generates a unique timestamp for use as an \texttt{<object-name>}~\footnote{For more examples of shell expansions see~\url{https://www.gnu.org/software/bash/manual/html\_node/Shell-Expansions.html}.}.

The first word~\footurl{https://www.gnu.org/software/bash/manual/html_node/Shell-Syntax.html} in a command is generally the callable~\ref{callables}. The rest of the command is expected to follow the conventions of the callable. In this paper we propose guidelines that we later demonstrate lead to useful expansions~\ref{expansions}.

\subsubsection{Callables}\label{callables}

A \emph{callable} is a valid command with no space and control operators~\cite{control_operators}. The list of callables depends on the machine's state and is generally extended through terraforming. Some of the well-known callables are \emph{git}~\cite{git}, \emph{docker}~\cite{docker}, \emph{pushd}~\cite{pushd}, \emph{nano}~\cite{nano}. 

\begin{theorem}
For any callable \texttt{<callable>}, and any string \texttt{<suffix>}, \texttt{<callable> <suffix>} is a valid command.
\end{theorem}

\begin{theorem}
For any valid command \texttt{<command>}, and any string \texttt{<suffix>}, \texttt{<command> <suffix>} is a valid command.
\end{theorem}

The \keyword{core} is a callable. Many \keyword{plugins} define their callable. 
\subsubsection{Command Templates}
\label{command-template}

Commands can be similar when considered as strings of characters. Here is a command that is similar to the above,
%
\begin{verbatim}
vanwatch ingest toronto upload . --count 3
\end{verbatim}
%
A \emph{command template} is a representation that yields similar commands, given the following rules. First, \placeholder{description} can be replaced with any string of characters that can be described as ``description''. See \emph{objects}~\ref{objects}, \emph{options}~\ref{options}, and \emph{arguments}~\ref{arguments}, for the next rules. Here is a command template for the two above,
%
\begin{verbatim}
vanwatch ingest \
    <area> \
    [dryrun,~upload] \
    [<object-name>] \
    [--count <-1>]
\end{verbatim}
