\documentclass{article}
\usepackage[english]{babel}

% Replace `letterpaper' with `a4paper' for UK/EU standard size
\usepackage[letterpaper,top=2cm,bottom=2cm,left=3cm,right=3cm,marginparwidth=1.75cm]{geometry}

\usepackage{amsmath}
\usepackage{graphicx}
\usepackage[colorlinks=true, allcolors=blue]{hyperref}

\newcommand{\revision}{1.96.1}

\newcommand{\footurl}[1]{\footnote{\url{#1}}}

\newcommand{\abcli}[1]{\url{https://github.com/kamangir/awesome-bash-cli/blob/2023-06-aws-batch-a/#1}}

\newcommand{\placeholder}[1]{$\langle\text{#1}\rangle$}

\title{$\nabla$ giza: A Recipe for AI Languages}
\author{Arash Abadpour - arash@abadpour.com}

\begin{document}
\maketitle

\begin{abstract}
In this paper, we discuss the Mathematics of building Machine Vision AI systems in Linux. We review the general challenge of translating the description of an AI operation in human language into a human-readable, machine-executable script. We select multiple Machine Vision AI challenges that we first describe in human language. Then, in each case, we build the language to convert the description in human language into one or more scripts we execute on machines. We use AWS SageMaker~\cite{aws_sage_maker} for development and training and AWS Batch~\cite{aws_batch} for inference and discuss API calls. The main contribution of this paper is a mathematical framework for building an AI language for a practical use-case in Machine Vision. We hope that researchers in others fields of AI use and extend this framework in their disciplines. We present a reference implementation of this framework as \emph{abcli}~\cite{abcli} - \emph{revision-\revision}
\end{abstract}

\tableofcontents

\section{Problem Definition}

wip

\section{Examples}
\subsection{Vancouver-Watching (vanwatch)}
\label{vanwatch}

\texttt{vanwatch} is a callable that,
%
\begin{enumerate}
    \item{terraforms the machine and the shell~\ref{conda}.}
    \item{discovers the cameras in an area~\ref{vanwatch_discover}.}
    \item{ingests images from the cameras discovered in an area~\ref{vanwatch_ingest}.}
    \item{detects the objects in the images ingested from an area and produces summary statistics~\ref{vanwatch_process}.}
\end{enumerate}

\subsubsection{vanwatch discover}
\label{vanwatch_discover}

Cameras are represented in different formats in different areas. \texttt{vanwatch discover area=<area>} discovers the cameras in \texttt{<area>} and stores them in \texttt{<area>.geojson} in the object \texttt{<object-name>}~\ref{objects} tagged~\ref{tag} for use for ingest~\ref{vanwatch_ingest}.

\begin{verbatim}
vanwatch discover \
    [area=<area>,~upload] \
    [-|<object-name>] \
    [<args>]
\end{verbatim}

\subsubsection{vanwatch ingest}
\label{vanwatch_ingest}

\texttt{vanwatch ingest area=<area>,count=<count> <object-name>} finds the latest set of cameras discovered~\ref{vanwatch_discover} in \texttt{<area>} through tag search~\ref{tag} and ingests \texttt{count} images into \texttt{<object-name>} and the runs \texttt{vanwatch process}~\ref{vanwatch_process} unless \texttt{\~process}.

\begin{verbatim}
vanwatch ingest \
    [area=<area>,count=<-1>,dryrun,model=<model-id>,~process,~upload] \
    [-|<object-name>] \
    <args>
\end{verbatim}

\subsubsection{vanwatch process}
\label{vanwatch_process}

\texttt{vanwatch process - <object-name>} runs object detection~\footurl{https://hub.ultralytics.com/models/R6nMlK6kQjSsQ76MPqQM?tab=preview} on the images ingested into \texttt{<object-name>} and updates \texttt{<area>.geojson}. \texttt{vanwatch process} reuses the inference already present in the object and completes the missing pieces.

\begin{verbatim}
vanwatch process \
    [~download,model=<model-id>,~upload] \
    [.|<object-name>] \
    [<args>]
\end{verbatim}


\bibliographystyle{acm}
\bibliography{giza}

\end{document}