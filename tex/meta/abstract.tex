\begin{quote}
    \enquote{... Do one thing well is usually interpreted as being about simplicity. But it's also, implicitly and at least as importantly, about orthogonality ...}~\cite{orthogonality}.
\end{quote}


\begin{abstract}

% Access, Automation, Analytics -> AI

Unrestricted \emph{Access} to algo and data is essential to augment the limited cognitive capabilities of humans, enabling the construction of vast systems that expand beyond the horizon. These large-scale systems are often developed by individuals who may never meet or might even be antagonistic due to local office dynamics. The next desires are \emph{Automation}, to enable linear descriptions of exponentially-expanding complexity, and \emph{Analytics}, to compress the resulting volumes of information into the bandwidth of human operators for effective consumption.

Examples in this paper attempt to demonstrate that Access, Automation, and Analytics are achieved through custom languages that are built for the context based on a mathematical model. This model, for the sake of demonstration, is comparable to fitting the spanning tree of an Application Programming Interface (API) to a dataset of use of AI tools by humans recorded as sequences of keystrokes and web interactions. We propose a mathematical framework for discussing this category of languages and provide a reference implementation based on \emph{Bash}~\cite{gnu_bash} expansions that call into \emph{Python}~\cite{python}. We note that Access, Automation, and Analytics have been suggested as the precursors of AI~\cite{Slife24}.

In this paper, we discuss the mathematics of building machine vision systems in Linux. We review the general challenge of translating the description of an AI operation in human language into a human-readable, machine-executable script. We select multiple machine vision challenges that we first describe in human language. Then, in each case, we build the language to convert the description in human language into one or more scripts we execute on machines. We use AWS SageMaker~\cite{sagemaker} for development and training and AWS Batch~\cite{aws_batch} for inference and compute. 

The main contribution of this paper is a mathematical framework for building an AI language for a practical use-case in machine vision. We hope that researchers in other fields of AI use and extend this framework in their disciplines. We present a reference implementation~\cite{abcli} and multiple use-cases~\cite{vanwatch}\footnote{built by \texttt{gizai-\revision}.}.
\end{abstract}