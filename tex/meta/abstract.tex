\begin{abstract}

Access to algo and data is critical to enable human beings' limited cognitive capacity to build systems that stretch beyond the horizons. These large scale systems are built by individuals whom never meet or may be hostile due to the local office politics. The next desires are Automation, to enable linear descriptions of exponentially-expanding complexity~\footnote{``\emph{Now, let's run these algos on those datasets, on combinations of the inputs, at a set of k values, while multiple variables vary within inter-related ranges.}''}, and Analytics, to compress volumes of information into the bandwidth of human operators for safe consumption over lifetimes.

Examples in this paper attempt to demonstrate that Access, Automation, and Analytics can be achieved through building custom languages that are context-specific. The process of generating one such language, for the sake of demonstration, is comparable to fitting a spanning tree to a dataset of use of AI tools by humans recorded as keystrokes. We propose a mathematical framework for discussing this category of languages and provide a reference implementation based on Bash expansions that call into Python. We note that Access, Automation, and Analytics are the precursors of AI.

In this paper, we discuss the Mathematics of building Machine Vision AI systems in Linux. We review the general challenge of translating the description of an AI operation in human language into a human-readable, machine-executable script. We select multiple Machine Vision AI challenges that we first describe in human language. Then, in each case, we build the language to convert the description in human language into one or more scripts we execute on machines.

We use AWS SageMaker~\footurl{https://aws.amazon.com/sagemaker/} for development and training and AWS Batch~\footurl{https://aws.amazon.com/batch/} for inference and compute. 

The main contribution of this paper is a mathematical framework for building an AI language for a practical use-case in Machine Vision. We hope that researchers in other fields of AI use and extend this framework in their disciplines. We present the reference implementation \emph{abcli}, short for \emph{awesome-bash-cli}~\cite{abcli} and multiple use-cases~\footnote{bird watching in downtown Vancouver with AI, \url{https://github.com/kamangir/vancouver-watching}, \emph{vancouver-watching}, \emph{vanwatch}.}~\footnote{revision~\revision}.
\end{abstract}