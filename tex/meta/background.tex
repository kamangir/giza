\section{Background and Context}

Almost five years ago, on Thursday, November 8, 2018, I acquired a Raspberry Pi~\footnote{\url{https://www.raspberrypi.org/}} on Amazon. Since then, my personal and professional lives have focused on Linux. Professionally, I do AI and, more recently, geospatial AI. In my personal life, I mix AI, cloud, and mathematics into minimal forms that seek survival~\footnote{\url{https://github.com/kamangir}}. Over the years, I have built a set of mechanisms for building AI systems that I will document in this paper. Therefore, this is an attempt to produce formal mathematical definitions for the AI mechanisms that I will collectively refer to as \emph{giza}. I seek to understand these mechanisms through this effort better to use them more optimally and along new dimensions.

This paper discusses concepts at the intersection of mathematics, software science, and computer science, and lacks scientific rigour in many places. I intend to push the practical development of this theory to fruition and hope to receive guidance along the way from experts in the field and solidify the theoretical foundations.