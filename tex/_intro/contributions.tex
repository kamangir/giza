\subsection{Contributions of This Work}\label{contributions}

This work, first, proposes several novel Bash expansions based on command substitution~\ref{command_substitution}. Then, using typed positional arguments, we propose \keyword{options} and \keyword{objects}. Then, we discuss argument injection as a suffix~\ref{expansions_param_injection} and \emph{prefixing}~\ref{prefixing} to transform \texttt{<command>} to \texttt{<prefix> <options> <command>}

Then, we discuss the \keyword{core}, which is the \kCallable that is \texttt{source}'d in a startup file~\cite{bash_startup_files}. The \texttt{core} loads the \keyword{plugins} that add branches to the syntax. We then discuss the \kSeed; the notion that code generates code that is transferred into another machine through the clipboard, a key, or a \texttt{scp}~\cite{scp}--style protocol to terraform the machine and run a command~\ref{commands}. \kStart is a necessity; the first intelligent command to start the day with. \kStart behaves according to the machine it runs on and other aspects of the state~\ref{machine}. On a MacBook, \texttt{@start} logs in and starts an \texttt{ssh} session to the default machine. On that machine, \texttt{@start} starts the docker container~\footnote{What \texttt{@start} does is decided by its immediate user; the tool is adapted to the tool user.}. We then discuss \kGit.

This work also contributes a set of conventions~\ref{conventions} that enable more effective use of the proposed expansions, such as \keywordd{<command> help}{help} and \keywordd{@init}{init}.