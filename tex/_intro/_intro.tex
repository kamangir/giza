\section{Introduction}
\label{intro}

Bash is a \enquote{Unix shell and command language first released in 1989 that has been used as the default login shell for most Linux distributions}~\footurl{https://en.wikipedia.org/wiki/Bash_(Unix_shell)}. A shell is a \enquote{macro processor that executes commands}~\footurl{https://www.gnu.org/software/bash/manual/html_node/What-is-a-shell_003f.html}, where \enquote{macro processor means functionality where text and symbols are expanded to create larger expressions} (same reference). There are seven kinds of expansions in Bash~\footurl{https://www.gnu.org/software/bash/manual/html_node/Shell-Expansions.html}.

\emph{Brace Expansion}~\footurl{https://www.gnu.org/software/bash/manual/html_node/Brace-Expansion.html} is the first and the quickest to explain,
%
\begin{verbatim}
 > bash$ echo a{d,c,b}e
ade ace abe
\end{verbatim}
%
\emph{Tilde Expansion}~\footurl{https://www.gnu.org/software/bash/manual/html_node/Tilde-Expansion.html} relates to words that begin with an unquoted tilde character (\textasciitilde). \emph{Parameter and Variable Expansion}~\footurl{https://www.gnu.org/software/bash/manual/html_node/Shell-Parameter-Expansion.html} enable the use of variables, as \texttt{\$\{variable\}}, as well as more elaborate pattern matching forms such as \texttt{\$\{parameter/\#pattern/string\}}. \emph{Command Substitution} \enquote{allows the output of a command to replace the command itself} ~\footurl{https://www.gnu.org/software/bash/manual/html_node/Command-Substitution.html}. \emph{Arithmetic expansion}~\footurl{https://www.gnu.org/software/bash/manual/html_node/Arithmetic-Expansion.html} enables arithmetic operations using the form \texttt{\$(( expression ))
} and \emph{Word Splitting}\footurl{https://www.gnu.org/software/bash/manual/html_node/Word-Splitting.html} governs the splitting of the command to words. Finally, \emph{Filename Expansion}~\footurl{https://www.gnu.org/software/bash/manual/html_node/Filename-Expansion.html} enables the familiar wildcard reference to filenames using `*' and `?'. In Section~\ref{expansions} we propose a set of relevant expansions to AI operations that are implemented using Python~\footurl{https://github.com/kamangir/awesome-bash-cli}.

This work, first, proposes several novel Bash expansions based on command substitution~\ref{command_substitution}. Then, using typed positional arguments, we propose \keyword{options} and \keyword{objects}. Then, we discuss argument injection as a suffix~\ref{expansions_param_injection} and \emph{prefixing}~\ref{prefixing} to transform \texttt{<command>} to \texttt{<prefix> <options> <command>}

Then, we discuss the \keyword{core}, which is the \keyword{callable} that is \texttt{source}'d in a startup file~\footurl{https://www.gnu.org/software/bash/manual/html_node/Bash-Startup-Files.html}. The \texttt{core} loads the \keyword{plugins} that add branches to the syntax, and \keyword{scripts} that implement the last mile. We then discuss the \keyword2{@seed}{seed}; the notion that code generates code that is transferred into another machine through the clipboard, a key, or a \texttt{scp}~\footurl{https://linux.die.net/man/1/scp}--style protocol to terraform the machine and run a command~\ref{commands}. \keyword{@start} is a necessity; the first intelligent command to start the day with. \texttt{@start} behaves according to the machine it runs on and other aspects of the state~\ref{machine}. On a MacBook, \texttt{@start} logs in and starts an \texttt{ssh} session to the default machine. On that machine, \texttt{@start} starts the docker container~\footnote{What \texttt{@start} does is decided by its immediate user; the tool is adapted to the tool user.}. We then discuss \keyword{@git}.

This work also contributes a set of conventions~\ref{conventions} that enable more effective use of the proposed expansions, such as\keyword2{<command> help}{help} and \keyword2{@init}{init}.