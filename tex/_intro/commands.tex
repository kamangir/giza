\subsection{Commands}\label{command}\label{commands}

We are interested in a special category of valid bash commands~\cite{shell_syntax} that start with a \keyword{callable}, continues with a prescribed sequence of identifiers, and end with \kArguments. In this convention, the type of each identifier is known based on the command until that identifier, and is one of the following,
%
\begin{itemize}
    \item a string picked from a known list, or an \emph{Enum}~\cite{python_eunum}, generally identifiying a \emph{task}.
    \item a \emph{value}, either numerical or a filename, for example.
    \item an \keyword{options}.
    \item an \keyword{object}.
\end{itemize}
%
This is an example command,
%
\begin{verbatim}
vanwatch ingest \
    area=vancouver,~batch,count=5,gif . \
    --count 12
\end{verbatim}
%
wherein, ``\cVanwatch'' is a \texttt{callable}, ``\texttt{ingest}'' is the \emph{task}, ``\texttt{area=vancouver,~batch,count=5,gif}'' is an \texttt{options}, ``\texttt{.}'' is an \texttt{object} pointer, and ``\texttt{--count 12}'' are the \texttt{arguments}.