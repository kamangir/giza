\section{Abstraction}\label{math}

\newcommand{\angled}[1]{\langle \text{#1} \rangle}

\begin{Rule}
actions are executed on machines through commands.
\end{Rule}

\begin{Rule}
objects contain arbitrary collections of files and folders. 
\end{Rule}

\begin{Rule}
objects persist and can be downloaded/uploaded, in full or selectively.
\end{Rule}

\begin{Rule}
many actions produce new objects or modify existing ones possibly using one or more existing objects. 
\end{Rule}

\begin{Rule}
objects can have properties, as in $\angled{object-name}.\angled{keyword}=\angled{value}$. while all properties can be set and get, there are two general groups of properties: tags can be searched whereas metadata cannot. value is a single string in tags whereas metadata values can be arbitrarily large and of more sophisticated types.
\end{Rule}


\begin{Rule}
some commands generate a string that can be used as $\$(\angled{command})\$$ in other commands. 
\end{Rule}

\begin{Rule}
commands follow this syntax:
\end{Rule}
%
\begin{eqnarray}
\angled{command}=@\angled{callable}~
\begin{bmatrix}
    \angled{options}\\
    \angled{value}\\
    \angled{object-reference}
\end{bmatrix}^N~
\begin{bmatrix}
    --\angled{keyword}~\angled{value}
\end{bmatrix}^*,
\end{eqnarray}
%
where $\angled{options}$ follows this syntax:
%
\begin{eqnarray}
\angled{options}=
\left[\angled{keyword}=\angled{value}\right]^*,\left[\angled{keyword}\right]^*,\left[\sim\angled{keyword}\right]^*,
\end{eqnarray}
%
and $\angled{object-reference}$ is either an object name or one of $.$, $..$, $...$, and so on, which refer to objects that are $@select$ed.

\begin{Rule}
$@help\angled{command}$ shows the syntax of a $\angled{command}$ and describes what it does.
\end{Rule}
