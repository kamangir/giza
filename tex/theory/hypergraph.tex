\marginpar{Hypergraph of Objects \& Commands}
A group of \emph{operators} maintain a growing space of \emph{commands} in a set of ~\emph{repositories}, using a system such as \emph{git}~\cite{git} and following a collective peer-reviewed \emph{pull-request}~\cite{pull_request} process. Each operator can access a set of \emph{machines} on a system such as \emph{SageMaker}~\cite{sagemaker} and create \emph{shells} on them to run commands to produce \emph{objects} which have an implicit or explicit value.

The objective of every operator is to increase the quality and quantity of the objects they generate. Therefore, the operators are interested in mechanisms that enable them to build the commands in ways that allow them to generate large quantities of objects that adhere to requirements that are known in the future from objects that may not exist yet. This is particularly important because commands run other commands and some objects are intermediaries in generating other objects.

As such, we model an AI system as an infinite \emph{hypergraph}~\cite{Hypergraph13} where the nodes are the objects and the edges are the commands. We note that this is a Hypergraph, and not a regular graph, because any edge can modify or create zero or more objects from zero or more existing objects. One subset of commands take in zero objects and generate zero objects. Instead, these commands modify the~\emph{state} of one or more machines or shells, and thus participate in the generation of the objects further down the line.

Whereas, a conventional hypergraph generalizes by allowing the edges to connect any two subsets of nodes, here, we are also interested in a second generalization that allows an edge to ``call'' other edges. An important example of this process is when an operator builds an algo (an edge) and then submits a command to \emph{AWS Batch}~\cite{aws_batch} that calls the algo 10,000 times on separate subsets of objects. Similarly, an operator may call a command that deploys the algo as an API that is called on a stream of incoming objects.

Hypergraphs have been used in the past to model parallel data structures~\cite{HK00}, for data mining~\cite{HBC07}, and clustering~\cite{BP09}. For a list of other relevant uses of hypergraphs refer to~\cite{Hypergraph13}.