\section{Conventions}\label{conventions}\label{core}\label{help}

Conventions augment and enable \keyword{expansions} or are found to be helpful.
%
\begin{itemize}
    \item \texttt{@help <command>} shows help about \keyword{command}.
\end{itemize}  

\subsection{\texttt{\$<core>\_<suffix>}}\label{awareness}

During initialization, the \keyword{core} and the \keyword{plugins} set a series of variables to harmonize paths and other \keyword{machine}--specific parameters to enable the same code to run on different machines simultaneously. 
%
\begin{itemize}
    \item{\texttt{\$<core>\_git\_<suffix>}: \keywordd{@git}{git}.}
    \item{\texttt{\$<core>\_<type>\_name[\_prev[n]]}: \keyword{select}ed objects.}
    \item{\texttt{\$<core>\_<type>\_path}: \keyword{select}ed object path.}
    \item{\texttt{\$<core>\_is\_<suffix>} describe the \keyword{machine}}
    \item{\texttt{\$<core>\_path\_<suffix>}: paths.}
\end{itemize}
%
An alias of \texttt{env} lists these variables.
%
\begin{verbatim}
@env [<keyword>]
\end{verbatim}

This mechanism is also relevant to functions~\cite{namespaces}.

\subsection{\texttt{dryrun}}\label{dryrun}

In \texttt{dryrun} mode, the low-cost and fast aspects of the command are executed, and the rest of the operation is logged instead. \texttt{@eval} is used for this operation. 

\subsection{\texttt{--<keyword> <value>}}
\label{expansions_param_injection}

wip
\marginpar{\keyword{--delim space}}
Command substitution~\ref{command_substitution} is useful in \keyword{for} loops and other usages where a delimited list of keywords is consumed.
%
\begin{verbatim}
    local object_name
    for object_name in $(<command-1> \
        --count 2 \
        --offset 1 \
        --delim space); do
        <command-2> $object_name <args>
    done
\end{verbatim}
%
We recommend using the three arguments \keyword{--count}, \keyword{--delim}, and \keyword{--offset}.

\subsection{\texttt{@init}}\label{init}

The following expansions initialize the \keyword{core}, and therefore all \keyword{plugins}, or \texttt{<plugin-name>}.
%
\begin{verbatim}
@init <options>

<plugin-name> init <options>
\end{verbatim}

\subsection{Plugins}\label{plugins}

A plugin generally defines one or more \keyword{callables}.

\subsection{\texttt{@start}}\label{start}

\begin{verbatim}
@start <options>
\end{verbatim}

\subsection{\texttt{@conda}}\label{conda}

\begin{verbatim}
@conda create [clone=<base>,name=<environment-name>,~recreate]

@conda exists [<environment-name>]

@conda list

@conda remove|rm [<environment-name>]
\end{verbatim}

\subsection{\texttt{@publish}}\label{publish}

\begin{verbatim}
@publish [extension=<png>,filename=<filename-1+filename-2>,randomize,tar] \
    [.|<object-name>]
\end{verbatim}
%
copies select content from \texttt{<object-name>} into a publicly accessible bucket.


