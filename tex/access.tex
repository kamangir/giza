\section{Access}

\marginpar{Operators}
A group of operators maintain a growing space of \keyword{command}s in a set of repositories, using a system such as \keyword{git} and following a collective peer-reviewed \emph{pull-request}~\cite{pull_request} process. Each operator can access a set of \keyword{machines} and create \keyword{shells} on them to run \keyword{commands} to produce \keyword{objects} which have an implicit or explicit value. The objective of every operator is to increase the value of the objects that they generate.

The operators act asynchronously while communicating with each other. Multiple operators may simultaneously use the same machine, and the same operator may simultaneously use multiple machines. Only one operator uses a shell at one time. Some machines are exogenous to this model, yet the operators can access their states in read or write modes. Cloud storage~\cite{aws_s3} and compute resources~\cite{aws_batch} are examples of these machines.

\marginpar{Machines}
A \emph{machine} is a state machine that is connected to many other machines and shares some of its state with them for read and write. A \emph{shell} is a stateful access mechanism to a machine that an \keyword{operator} uses to run \keyword{commands}. Running a command in a shell can potentially modify the state of all other machines. Machines and shells may be restarted through running commands. After a restart, machines and shells maintain some of their state. Two examples of machines are a Raspberry Pi~\cite{rpi} that runs Linux and is connected to the AWS infrastructure~\cite{aws} and a docker container~\cite{docker} running in AWS Batch~\cite{aws_batch}. GNU Bash~\cite{gnu_bash} is an example of a shell.

\marginpar{Objects}
Data within the system is accessed as \emph{objects}. An object is a uniquely named collection of files and folders and can be downloaded or uploaded, in part or as a whole. Some objects already exist in the environment. For example, an item in a STAC collection (a datacube) or a dataset in kaggle are objects. The aim of the system is to produce high value objects, potentially using other objects. A curated dataset, a model trained on it, and the model's predictions on a datacube, are all objects.

\marginpar{Commands}
Objects are produced through running \emph{commands}. A command is a string of characters that is meaningful to Bash~\cite{gnu_bash}. Bash is a \enquote{Unix shell and command language first released in 1989 that has been used as the default login shell for most Linux distributions}~\cite{bash}. A shell is a \enquote{macro processor that executes commands}~\cite{bash_manual}, where \enquote{macro processor means functionality where text and symbols are expanded to create larger expressions}~\cite{bash_manual}. There are seven kinds of expansions~\cite{bash_expansions_ref} in Bash.

\marginpar{Expansions}
\emph{Brace Expansion}~\cite{brace_expansion} expands `a{d,c,b}e' to `ade ace abe'. \emph{Tilde Expansion}~\cite{tilde_expansion} relates to words that begin with an unquoted tilde character (\textasciitilde). \emph{Parameter and Variable Expansion}~\cite{shell_parameter_expansion} enable the use of variables, as \texttt{\$\{variable\}}, as well as more elaborate pattern matching forms such as \texttt{\$\{parameter/\#pattern/string\}}. \emph{Command Substitution} \enquote{allows the output of a command to replace the command itself} ~\cite{command_substitution}. \emph{Arithmetic expansion}~\cite{arithmetic_expansion} enables arithmetic operations using the form \texttt{\$(( expression ))} and \emph{Word Splitting}~\cite{word_splitting} governs the splitting of the command to words. Finally, \emph{Filename Expansion}~\cite{filename_expansion} enables the familiar wildcard reference to filenames using `*' and `?'.

\marginpar{Callables}
A \emph{callable} is a valid command with no space and control operators~\cite{control_operators}. Some of the well-known callables are \emph{git}~\cite{git}, \emph{docker}~\cite{docker}, \emph{pushd}~\cite{pushd}, and \emph{nano}~\cite{nano}. 

\marginpar{Syntax}
We are interested in a special category of valid bash commands~\cite{shell_syntax} that start with a \keyword{callable}, continues with a prescribed sequence of identifiers, and end with \kArguments. In this convention, the type of each identifier is known based on the command until that identifier, and is one of the following,
%
\begin{itemize}
    \item a string picked from a known list, or an \emph{Enum}~\cite{python_eunum}, generally identifiying a \emph{task}.
    \item a \emph{value}, either numerical or a filename, for example.
    \item an \keyword{options}.
    \item an \keyword{object}.
\end{itemize}
%
This is an example command,
%
\begin{verbatim}
vanwatch ingest \
    area=vancouver,~batch,count=5,gif . \
    --count 12
\end{verbatim}
%
wherein, ``\cVanwatch'' is a \texttt{callable}, ``\texttt{ingest}'' is the \emph{task}, ``\texttt{area=vancouver,~batch,count=5,gif}'' is an \texttt{options}, ``\texttt{.}'' is an \texttt{object} pointer, and ``\texttt{--count 12}'' are the \texttt{arguments}.


wip: Debug Turnaround